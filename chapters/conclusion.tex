\chapter{Conclusion}
\label{ch:conclusion}
The goal of the thesis was to implement a detection algorithm for MBZIRC 2020 contest. More specifically, the algorithm should detect the model of bricks for the second challenge of the contest. The Velodyne VLP-16 lidar sensor was used for the detection. 

A simple line extraction algorithm was applied to find individual bricks. At the beginning of the challenge the bricks were stacked into the piles, so we applied the EM algorithm to find parameters of individual piles. Further, the piles were always placed with fixed relative positions. Since all the piles together are too large objects for detection in a single measurement, it was necessary to create a global model. In our case, the global model was the symbolic map.

When the robot's localization was used, and all detections were properly transformed and sent to the symbolic map, the initial position of all piles had to be found. We proposed a version of the EM algorithm which can exploit the spatial distribution of a model. Parameters of such a model were found based on partial detections and their confidences. The last type of detection used the RANSAC algorithm to fit individual brick positions to the hypothesis.

There could be no gap between the bricks when the robot successfully placed them. That makes it impossible to distinguish individual bricks by the lidar data. The color camera images were used to color the pointcloud. After coloring, it was possible to use custom clustering and detect individual bricks, that are very close to each other.

In the end, we experimented on the data from the contest, and we have shown that all algorithms work. Although the line extraction algorithm produces a large number of false positives, it is sufficient to detect initial positions of all piles and the UAV destination wall. Different ground segmentation methods were discussed to reduce the number of false-positive measurements. We have also shown that other types of detection have a lower range, but they produce a significantly lower number of false-positive measurements. To sum up, all tasks of the thesis were successfully completed.