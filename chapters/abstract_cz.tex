\vfill
\begin{center}
{\it \large Abstrakt}
\vspace{0.2cm}

\begin{minipage}{0.8\textwidth}{
Soutěž Mohamed Bin Zayed International Robotic Challenge (MBZIRC) se zaměřuje na použití autonomních multi-robotických systémů pro úkoly, které jsou motivovány problémy reaálného světa. Jedním z úkolů bylo využít skupinu dronů a pozemní roboty pro postavení zdi z cihel. Jelikož všichni roboti museli pracovat plně autonomně, bylo naprosto nezbytné navrhnout systém, který může cihly spolihlivě detekovat. Diplomová práce popisuje metody, které byly použity pro detekci těchto cihel. Většina metod je založena na zpracování dat z lidaru, který nese pozemní robot. Paprsky z lidaru jsou zpracovány algoritmem split and merge, což umožňuje nalézt pozice cihel ve velmi řídkém mraku bodů. Dále je použit algoritmus RANSAC, který z malé vzdálennosti může ověřit vzájemné pozice různých druhů cihel. Kromě toho byl udělán odhad maximální věrohodnosti pro model hromady cihel, který je využit v inovativní aplikaci EM algoritmu. Pokud je použit správný model je takto možné nalézt jakýkoliv velký prostorově rozložený objekt pouze na základě částečných měření, s rozdílnou úrovní jistoty, uložených do symbolické mapy.
}
\end{minipage}
\end{center}
\vfill
\vspace{1cm}
\newpage{}
