\vfill
\begin{center}
{\it \large Abstract}
\vspace{0.2cm}

\begin{minipage}{0.8\textwidth}{
The MBZIRC contests are focused on using autonomous multi-robot systems for tasks that are motivated by real-world problems. One of the tasks was to exploit the group of drones and ground robots to build a wall from bricks. Because all robots have to operate autonomously, it is crucial to have a system that can reliably detect the bricks. The set of methods that were used to detect these bricks is described in the thesis. Most of the methods are based on the lidar data measured by the ground robot. The lidar scans are processed by the split and merge algorithm to find the bricks in a very sparse pointcloud. Further, the RANSAC algorithm is applied from close range to verify the spatial distribution of different types of bricks. Moreover, the maximum likelihood estimate of the brick pile model, for usage in the EM algorithm, is derived. With a proper model and symbolic map, it is possible to poll previously saved partial measurements with different levels of confidence and obtain a position of any large spatially distributed object.
\vspace{3mm}
\par \textbf{Keywords:} MBZIRC, object detection, lidar, pointcloud, hypothesis fitting, EM algorithm, RANSAC
}
\end{minipage}
\end{center}
\vfill
\vspace{1cm}
